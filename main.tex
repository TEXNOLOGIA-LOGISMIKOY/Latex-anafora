\documentclass{article}
\usepackage{graphicx}
\usepackage{hyperref}

\title{Data Analysis Application Report}
\author{Πέτρος Βεζυργιανόπουλος, Άγγελος Τσάνη}
\date{\today}

\begin{document}

\maketitle

\section{Introduction}
This report documents the functionalities of the Data Analysis Application, designed for data mining and analysis. The application allows users to upload data, perform two-dimensional visualizations, conduct exploratory data analysis, and compare classification and clustering algorithms.

\section{Data Upload}
Users can upload CSV or Excel files. The data is then displayed in a table for initial inspection.

\section{Two-Dimensional Visualization}
The application provides two methods for 2D visualization:
\begin{itemize}
    \item Principal Component Analysis (PCA)
    \item t-Distributed Stochastic Neighbor Embedding (t-SNE)
\end{itemize}
These methods reduce the dimensionality of the data to two dimensions for visualization purposes.

\section{Exploratory Data Analysis (EDA)}
Users can perform EDA using:
\begin{itemize}
    \item Histograms
    \item Boxplots
\end{itemize}
These plots help in understanding the distribution and variance of numerical features in the dataset.

\section{Classification Comparison}
The application compares the performance of two classification algorithms:
\begin{itemize}
    \item Logistic Regression
    \item Random Forest
\end{itemize}
Users can adjust the regularization parameter for Logistic Regression and the number of estimators for Random Forest. The accuracy of each classifier is displayed after running the comparison.

\section{Clustering Comparison}
The application supports clustering with:
\begin{itemize}
    \item K-Means
    \item Hierarchical Clustering
\end{itemize}
Users can specify the number of clusters for each algorithm. The cluster labels are displayed for both methods after running the comparison.

\section{Implementation Details}
The application is implemented using:
\begin{itemize}
    \item \textbf{Streamlit} for the web interface
    \item \textbf{Pandas} for data manipulation
    \item \textbf{Plotly} for visualizations
    \item \textbf{Scikit-learn} for machine learning algorithms
\end{itemize}

\section{Team}
\begin{itemize}
    \item Πέτρος Βεζυργιανόπουλος
    \item Άγγελος Τσάνη
\end{itemize}

\section{Tasks Executed}
\begin{itemize}
    \item 2D Visualization
    \item Docker setup
    \item Github version control
    \item Machine learning algorithms comparison
    \item Software release life cycle
\end{itemize}

\section{Conclusion}
This application provides a comprehensive tool for data analysis, enabling users to visualize, analyze, and compare data using various methods. The integration of different machine learning algorithms and visualization techniques offers a robust platform for exploratory data analysis and model evaluation.

\end{document}